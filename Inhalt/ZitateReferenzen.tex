\chapter{Zitate und Referenzen}
\label{cha:ZitateReferenzen}

Die \NeuerBegriff{Service-orientierte Architektur} (SOA) ist seit einiger Zeit \textit{das} Schlagwort im Bereich der Informationstechnologie. So haben \zB Deutschlands gr��te Softwarehersteller SAP und die Software AG ihre Unternehmensstrategie komplett auf die SOA ausgerichtet. \Autor{SAP2007} bietet mit \Fachbegriff{Netweaver} seine marktf�hrende ERP-Software auf Basis von SOA an,\footnote{\Vgl\Zitat[S.~127]{SAP2007}} und die \Autor{Software2007b}, die sich selbst als "`The XML Company"' bezeichnet, erweiterte k�rzlich noch einmal ihr bereits durchg�ngig an der SOA orientiertes Produktportfolio durch den Kauf des amerikanischen Unternehmens webMethods um L�sungen zur Unterst�tzung von Gesch�ftsprozessen.\footnote{\Vgl\Zitat{Software2007b}} In einem Atemzug mit der SOA werden h�ufig Webservices genannt, da sie durch ihre hohe Plattformunabh�ngigkeit und den Einsatz von Internettechnologie oftmals als Referenzimplementierung f�r die Services in einer SOA angef�hrt werden. Doch welche Vorteile bietet der Einsatz von Webservices in Unternehmen? K�nnen mit ihnen tats�chlich flexiblere Softwaresysteme entwickelt werden? Und wie einfach ist die Implementierung von Webservices auf unterschiedlichen Plattformen? Diesen Fragen wird sich der Autor in der vorliegenden Arbeit widmen.

Wie bereits in Kapitel \ref{cha:Einleitung} auf Seite \pageref{cha:Einleitung} erw�hnt, ist zur Unterst�tzung von Gesch�ftsprozessen der Einsatz von Informationstechnologie notwendig. Der Autor verfolgt mit dieser Arbeit das Ziel, einen Gesch�ftsprozess \todo{Was ist ein Gesch�ftsprozess?} mit Hilfe von Webservices zu optimieren. Hierzu wird er in diesem Kapitel eine Einf�hrung in das Thema Webservices und die damit in Zusammenhang stehenden Technologien geben, und auch auf m�gliche Einsatzbereiche von Webservices im Rahmen der Gesch�ftsprozessoptimierung eingehen. Tabelle \ref{tab:ElementeDerEreignisgesteuertenProzesskette} auf Seite \pageref{tab:ElementeDerEreignisgesteuertenProzesskette} zeigt ganz tolle Sachen.

Ich empfehle allen Softwareentwicklern die Lekt�re von \Zitat{Goodliffe2007}.

\section{Definitionen}
Die Service-orientierte Architektur ist ein Ansatz der Softwareentwicklung, der sich stark am Konzept der Gesch�ftsprozesse orientiert und mit Hilfe von Webservices implementiert werden kann. In den beiden folgenden Kapiteln werden beide Begriffe eingehend erl�utert, worauf in Kapitel \ref{cha:Fazit} die f�r die Umsetzung von Webservices ben�tigten Technologien vorgestellt werden.

\section{Service-orientierte Architektur}
\Autor{OASIS2007}\footnote{Die \NeuerBegriff{Organization for the Advancement of Structured Information Standards} ist nach \Zitat{OASIS2007} ein internationales Konsortium aus �ber 600 Organisationen, das sich der Entwicklung von E-Business-Standards verschrieben hat. Mitglieder sind \zB IBM, SAP und Sun.} definiert den Begriff \NeuerBegriff{Service-orientierte Architektur} (SOA) wie folgt:
\begin{quote}
"`\textbf{Service Oriented Architecture} [\ldots] is a paradigm for organizing and utilizing distributed \textbf{capabilities} that may be under the control of different ownership domains."'\footnote{\Zitat[S.~8]{OASIS2006a}}
\end{quote}
Diese bewusst allgemein gehaltene Definition stammt aus dem Referenzmodell der SOA aus dem Jahr 2006. Dieses Modell wurde mit dem Ziel entwickelt, ein einheitliches Verst�ndnis des Begriffs SOA und des verwendeten Vokabulars zu schaffen, und sollte die zahlreichen bis dato vorhandenen, teils widerspr�chlichen Definitionen abl�sen.\footnote{\Vgl\Zitat[S.~4]{OASIS2006a}} Dabei wird zun�chst noch kein Bezug zur Informationstechnologie hergestellt, sondern allgemein von F�higkeiten gesprochen, die Personen, Unternehmen, aber eben auch Computer besitzen und evtl. Anderen anbieten, um Probleme zu l�sen. Als Beispiel wird ein Energieversorger angef�hrt, der Haushalten seine F�higkeit Strom zu erzeugen anbietet.\footnote{\Vgl\Zitat[S.~8f.]{OASIS2006a}}
