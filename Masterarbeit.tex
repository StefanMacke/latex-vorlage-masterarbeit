% ------------------------------------------------------------------------------
% Formatvorlage f�r Masterarbeiten an der Ohm-Hochschule N�rnberg
% ------------------------------------------------------------------------------
%   erstellt von Stefan Macke, 24.04.2009
%   http://blog.stefan-macke.de

% Dokumentenkopf ---------------------------------------------------------------
%   Diese Vorlage basiert auf "scrreprt" aus dem koma-script.
% ------------------------------------------------------------------------------
\documentclass[
    11pt, % Schriftgr��e
    DIV10,
    ngerman, % f�r Umlaute, Silbentrennung etc.
    a4paper, % Papierformat
    oneside, % einseitiges Dokument
    titlepage, % es wird eine Titelseite verwendet
    parskip=half, % Abstand zwischen Abs�tzen (halbe Zeile)
    headings=normal, % Gr��e der �berschriften verkleinern
    listof=totoc, % Verzeichnisse im Inhaltsverzeichnis auff�hren
    bibliography=totoc, % Literaturverzeichnis im Inhaltsverzeichnis auff�hren
    index=totoc, % Index im Inhaltsverzeichnis auff�hren
    captions=tableheading, % Beschriftung von Tabellen unterhalb ausgeben
    final % Status des Dokuments (final/draft)
]{scrreprt}

% Meta-Informationen -----------------------------------------------------------
%   Informationen �ber das Dokument, wie z.B. Titel, Autor, Matrikelnr. etc
%   werden in der Datei Meta.tex definiert und k�nnen danach global
%   verwendet werden.
% ------------------------------------------------------------------------------
% Meta-Informationen -----------------------------------------------------------
%   Definition von globalen Parametern, die im gesamten Dokument verwendet
%   werden k�nnen (z.B auf dem Deckblatt etc.).
%
%   ACHTUNG: Wenn die Texte Umlaute oder ein Esszet enthalten, muss der folgende
%            Befehl bereits an dieser Stelle aktiviert werden:
%            \usepackage[latin1]{inputenc}
% ------------------------------------------------------------------------------
\newcommand{\titel}{Titel der Masterarbeit}
\newcommand{\untertitel}{und hier kommt der Untertitel}
\newcommand{\art}{Masterarbeit}
\newcommand{\fachgebiet}{Software-Engineering}
\newcommand{\autor}{Hans Meier}
\newcommand{\studienbereich}{Software-Engineering}
\newcommand{\matrikelnr}{12 34 56}
\newcommand{\erstgutachter}{Prof. Dr. Werner Berentzen}
\newcommand{\zweitgutachter}{Dipl.-Inf. Lukas Podolski}
\newcommand{\jahr}{2009}
\newcommand{\ort}{Berlin}
\newcommand{\logo}{LogoOhmHochschule.jpg}


% ben�tigte Packages -----------------------------------------------------------
%   LaTeX-Packages, die ben�tigt werden, sind in die Datei Packages.tex
%   "ausgelagert", um diese Vorlage m�glichst �bersichtlich zu halten.
% ------------------------------------------------------------------------------
% Anpassung des Seitenlayouts --------------------------------------------------
%   siehe Seitenstil.tex
% ------------------------------------------------------------------------------
\usepackage[
    automark, % Kapitelangaben in Kopfzeile automatisch erstellen
    headsepline, % Trennlinie unter Kopfzeile
    ilines % Trennlinie linksb�ndig ausrichten
]{scrpage2}

% Anpassung an Landessprache ---------------------------------------------------
\usepackage[ngerman]{babel}

% Umlaute ----------------------------------------------------------------------
%   Umlaute/Sonderzeichen wie ���� direkt im Quelltext verwenden (CodePage).
%   Erlaubt automatische Trennung von Worten mit Umlauten.
% ------------------------------------------------------------------------------
\usepackage[latin1]{inputenc}
\usepackage[T1]{fontenc}
\usepackage{textcomp} % Euro-Zeichen etc.

% Schrift ----------------------------------------------------------------------
\usepackage{lmodern} % bessere Fonts
\usepackage{relsize} % Schriftgr��e relativ festlegen

% Grafiken ---------------------------------------------------------------------
% Einbinden von JPG-Grafiken erm�glichen
\usepackage[dvips,final]{graphicx}
% hier liegen die Bilder des Dokuments
\graphicspath{{Bilder/}}

% Befehle aus AMSTeX f�r mathematische Symbole z.B. \boldsymbol \mathbb --------
\usepackage{amsmath,amsfonts}

% f�r Index-Ausgabe mit \printindex --------------------------------------------
\usepackage{makeidx}

% Einfache Definition der Zeilenabst�nde und Seitenr�nder etc. -----------------
\usepackage{setspace}
\usepackage{geometry}

% Symbolverzeichnis ------------------------------------------------------------
%   Symbolverzeichnisse bequem erstellen. Beruht auf MakeIndex:
%     makeindex.exe %Name%.nlo -s nomencl.ist -o %Name%.nls
%   erzeugt dann das Verzeichnis. Dieser Befehl kann z.B. im TeXnicCenter
%   als Postprozessor eingetragen werden, damit er nicht st�ndig manuell
%   ausgef�hrt werden muss.
%   Die Definitionen sind ausgegliedert in die Datei "Glossar.tex".
% ------------------------------------------------------------------------------
\usepackage[intoc]{nomencl}
\let\abbrev\nomenclature
\renewcommand{\nomname}{Abk�rzungsverzeichnis}
\setlength{\nomlabelwidth}{.25\hsize}
\renewcommand{\nomlabel}[1]{#1 \dotfill}
\setlength{\nomitemsep}{-\parsep}

% zum Umflie�en von Bildern ----------------------------------------------------
\usepackage{floatflt}


% zum Einbinden von Programmcode -----------------------------------------------
\usepackage{listings}
\usepackage{xcolor} 
\definecolor{hellgelb}{rgb}{1,1,0.9}
\definecolor{colKeys}{rgb}{0,0,1}
\definecolor{colIdentifier}{rgb}{0,0,0}
\definecolor{colComments}{rgb}{1,0,0}
\definecolor{colString}{rgb}{0,0.5,0}
\lstset{
    float=hbp,
    basicstyle=\ttfamily\color{black}\small\smaller,
    identifierstyle=\color{colIdentifier},
    keywordstyle=\color{colKeys},
    stringstyle=\color{colString},
    commentstyle=\color{colComments},
    columns=flexible,
    tabsize=2,
    frame=single,
    extendedchars=true,
    showspaces=false,
    showstringspaces=false,
    numbers=left,
    numberstyle=\tiny,
    breaklines=true,
    backgroundcolor=\color{hellgelb},
    breakautoindent=true
}

% URL verlinken, lange URLs umbrechen etc. -------------------------------------
\usepackage{url}

% wichtig f�r korrekte Zitierweise ---------------------------------------------
\usepackage[square]{natbib}

% PDF-Optionen -----------------------------------------------------------------
\usepackage[
    bookmarks,
    bookmarksopen=true,
    colorlinks=true,
% diese Farbdefinitionen zeichnen Links im PDF farblich aus
    linkcolor=red, % einfache interne Verkn�pfungen
    anchorcolor=black,% Ankertext
    citecolor=blue, % Verweise auf Literaturverzeichniseintr�ge im Text
    filecolor=magenta, % Verkn�pfungen, die lokale Dateien �ffnen
    menucolor=red, % Acrobat-Men�punkte
    urlcolor=cyan, 
% diese Farbdefinitionen sollten f�r den Druck verwendet werden (alles schwarz)
    %linkcolor=black, % einfache interne Verkn�pfungen
    %anchorcolor=black, % Ankertext
    %citecolor=black, % Verweise auf Literaturverzeichniseintr�ge im Text
    %filecolor=black, % Verkn�pfungen, die lokale Dateien �ffnen
    %menucolor=black, % Acrobat-Men�punkte
    %urlcolor=black, 
    backref,
    plainpages=false, % zur korrekten Erstellung der Bookmarks
    pdfpagelabels, % zur korrekten Erstellung der Bookmarks
    hypertexnames=false, % zur korrekten Erstellung der Bookmarks
    linktocpage % Seitenzahlen anstatt Text im Inhaltsverzeichnis verlinken
]{hyperref}
% Befehle, die Umlaute ausgeben, f�hren zu Fehlern, wenn sie hyperref als Optionen �bergeben werden
\hypersetup{
    pdftitle={\titel \untertitel},
    pdfauthor={\autor},
    pdfcreator={\autor},
    pdfsubject={\titel \untertitel},
    pdfkeywords={\titel \untertitel},
}

% fortlaufendes Durchnummerieren der Fu�noten ----------------------------------
\usepackage{chngcntr}

% f�r lange Tabellen -----------------------------------------------------------
\usepackage{longtable}
\usepackage{array}
\usepackage{ragged2e}
\usepackage{lscape}

% Spaltendefinition rechtsb�ndig mit definierter Breite ------------------------
\newcolumntype{w}[1]{>{\raggedleft\hspace{0pt}}p{#1}}

% Formatierung von Listen �ndern -----------------------------------------------
\usepackage{paralist}

% bei der Definition eigener Befehle ben�tigt
\usepackage{ifthen}

% definiert u.a. die Befehle \todo und \listoftodos
\usepackage{todonotes}

% sorgt daf�r, dass Leerzeichen hinter parameterlosen Makros nicht als Makroendezeichen interpretiert werden
\usepackage{xspace}


% Erstellung eines Index und Abk�rzungsverzeichnisses aktivieren ---------------
\makeindex
\makenomenclature

% Kopf- und Fu�zeilen, Seitenr�nder etc. ---------------------------------------
% Zeilenabstand 1,5 Zeilen -----------------------------------------------------
\onehalfspacing

% Seitenr�nder -----------------------------------------------------------------
\setlength{\topskip}{\ht\strutbox} % behebt Warnung von geometry
\geometry{paper=a4paper,left=35mm,right=35mm,top=30mm}

% Kopf- und Fu�zeilen ----------------------------------------------------------
\pagestyle{scrheadings}
% Kopf- und Fu�zeile auch auf Kapitelanfangsseiten
\renewcommand*{\chapterpagestyle}{scrheadings} 
% Schriftform der Kopfzeile
\renewcommand{\headfont}{\normalfont}

% Kopfzeile
\ihead{\large{\textsc{\titel}}\\ \small{\untertitel} \\[2ex] \textit{\headmark}}
\chead{}
\ohead{\includegraphics[scale=0.15]{\logo}}
\setlength{\headheight}{21mm} % H�he der Kopfzeile
% Kopfzeile �ber den Text hinaus verbreitern
\setheadwidth[0pt]{textwithmarginpar} 
\setheadsepline[text]{0.4pt} % Trennlinie unter Kopfzeile

% Fu�zeile
\ifoot{\copyright\ \autor}
\cfoot{}
\ofoot{\pagemark}

% sonstige typographische Einstellungen ----------------------------------------

% erzeugt ein wenig mehr Platz hinter einem Punkt
\frenchspacing 

% Schusterjungen und Hurenkinder vermeiden
\clubpenalty = 10000
\widowpenalty = 10000 
\displaywidowpenalty = 10000

% Quellcode-Ausgabe formatieren
\lstset{numbers=left, numberstyle=\tiny, numbersep=5pt, breaklines=true}
\lstset{emph={square}, emphstyle=\color{red}, emph={[2]root,base}, emphstyle={[2]\color{blue}}}

% Fu�noten fortlaufend durchnummerieren
\counterwithout{footnote}{chapter}


% eigene Definitionen f�r Silbentrennung
\include{Silbentrennung}

% eigene LaTeX-Befehle
% Eigene Befehle und typographische Auszeichnungen f�r diese

% einfaches Wechseln der Schrift, z.B.: \changefont{cmss}{sbc}{n}
\newcommand{\changefont}[3]{\fontfamily{#1} \fontseries{#2} \fontshape{#3} \selectfont}

% Abk�rzungen mit korrektem Leerraum 
\newcommand{\ua}{\mbox{u.\,a.\ }}
\newcommand{\zB}{\mbox{z.\,B.\ }}
\newcommand{\dahe}{\mbox{d.\,h.\ }}
\newcommand{\Vgl}{Vgl.\ }
\newcommand{\bzw}{bzw.\ }
\newcommand{\evtl}{evtl.\ }

\newcommand{\abbildung}[1]{Abbildung~\ref{fig:#1}}

\newcommand{\bs}{$\backslash$}

% erzeugt ein Listenelement mit fetter �berschrift 
\newcommand{\itemd}[2]{\item{\textbf{#1}}\\{#2}}

% einige Befehle zum Zitieren --------------------------------------------------
\newcommand{\Zitat}[2][\empty]{\ifthenelse{\equal{#1}{\empty}}{\citep{#2}}{\citep[#1]{#2}}}

% zum Ausgeben von Autoren
\newcommand{\AutorName}[1]{\textsc{#1}}
\newcommand{\Autor}[1]{\AutorName{\citeauthor{#1}}}

% verschiedene Befehle um W�rter semantisch auszuzeichnen ----------------------
\newcommand{\NeuerBegriff}[1]{\textbf{#1}}
\newcommand{\Fachbegriff}[1]{\textit{#1}}

\newcommand{\Eingabe}[1]{\texttt{#1}}
\newcommand{\Code}[1]{\texttt{#1}}
\newcommand{\Datei}[1]{\texttt{#1}}

\newcommand{\Datentyp}[1]{\textsf{#1}}
\newcommand{\XMLElement}[1]{\textsf{#1}}
\newcommand{\Webservice}[1]{\textsf{#1}}


% Das eigentliche Dokument -----------------------------------------------------
%   Der eigentliche Inhalt des Dokuments beginnt hier. Die einzelnen Seiten
%   und Kapitel werden in eigene Dateien ausgelagert und hier nur inkludiert.
% ------------------------------------------------------------------------------
\begin{document}

% auch subsubsection nummerieren
\setcounter{secnumdepth}{3}
\setcounter{tocdepth}{3}

% Deckblatt und Abstract ohne Seitenzahl
\ofoot{}
\thispagestyle{plain}
\begin{titlepage}

\begin{center}

\huge{\textbf{\titel}}\\[1.5ex]
\LARGE{\textbf{\untertitel}}\\[6ex]
\LARGE{\textbf{\art}}\\[1.5ex]
\Large{im Fachgebiet \fachgebiet}\\[18ex]

\includegraphics[scale=0.2]{LogoOhmHochschuleMitText.jpg}\\[6ex]

\normalsize
\begin{tabular}{w{5.4cm}p{6cm}}\\
vorgelegt von:  & \quad \autor\\[1.2ex]
Studienbereich: & \quad \studienbereich\\[1.2ex]
Matrikelnummer: & \quad \matrikelnr\\[1.2ex]
Erstgutachter:  & \quad \erstgutachter\\[1.2ex]
Zweitgutachter: & \quad \zweitgutachter\\[3ex]
\end{tabular}

\copyright\ \jahr\\[9ex]

\end{center}

\singlespacing
\small
\noindent Dieses Werk einschlie�lich seiner Teile ist \textbf{urheberrechtlich gesch�tzt}. Jede Verwertung au�erhalb der engen Grenzen des Urheberrechtgesetzes ist ohne Zustimmung des Autors unzul�ssig und strafbar. Das gilt insbesondere f�r Vervielf�ltigungen, �bersetzungen, Mikroverfilmungen sowie die Einspeicherung und Verarbeitung in elektronischen Systemen.

\end{titlepage}

\include{Inhalt/Abstract}
\ofoot{\pagemark}

% Seitennummerierung -----------------------------------------------------------
%   Vor dem Hauptteil werden die Seiten in gro�en r�mischen Ziffern 
%   nummeriert.
% ------------------------------------------------------------------------------
\pagenumbering{Roman}
\tableofcontents % Inhaltsverzeichnis

% Abk�rzungsverzeichnis --------------------------------------------------------
\nomenclature{API}{Application Programming Interface}
\nomenclature{ARIS}{Architektur integrierter Informationssysteme}
\nomenclature{BPR}{Business Process Reengineering}
\nomenclature{eEPK}{erweiterte Ereignisgesteuerte Prozesskette}
\nomenclature{EPK}{Ereignisgesteuerte Prozesskette}
\nomenclature{JMS}{Java Message Service}
\nomenclature{SDK}{Software Development Kit}
\nomenclature{URI}{Uniform Resource Identifier}
\nomenclature{URL}{Uniform Resource Locator}
\nomenclature{URN}{Uniform Resource Name}
\nomenclature{W3C}{World Wide Web Consortium}
\nomenclature{XML}{Extensible Markup Language}
\nomenclature{XPath}{XML Path Language}
\nomenclature{XSL}{Extensible Stylesheet Language}
\nomenclature{XSLT}{XSL Transformations}

% f�r korrekte �berschrift in der Kopfzeile
\clearpage\markboth{\nomname}{\nomname} 
\printnomenclature
\label{sec:Glossar}

\listoffigures % Abbildungsverzeichnis
\listoftables % Tabellenverzeichnis
\renewcommand{\lstlistlistingname}{Verzeichnis der Listings}
\lstlistoflistings % Listings-Verzeichnis

% arabische Seitenzahlen im Hauptteil ------------------------------------------
\clearpage
\pagenumbering{arabic}

% die Inhaltskapitel werden in "Inhalt.tex" inkludiert -------------------------
% Hier k�nnen die einzelnen Kapitel inkludiert werden. Sie m�ssen in den 
% entsprechenden .TEX-Dateien vorliegen. Die Dateinamen k�nnen nat�rlich 
% angepasst werden.

\include{Inhalt/Einleitung}
\chapter{Zitate und Referenzen}
\label{cha:ZitateReferenzen}

Die \NeuerBegriff{Service-orientierte Architektur} (SOA) ist seit einiger Zeit \textit{das} Schlagwort im Bereich der Informationstechnologie. So haben \zB Deutschlands gr��te Softwarehersteller SAP und die Software AG ihre Unternehmensstrategie komplett auf die SOA ausgerichtet. \Autor{SAP2007} bietet mit \Fachbegriff{Netweaver} seine marktf�hrende ERP-Software auf Basis von SOA an,\footnote{\Vgl\Zitat[S.~127]{SAP2007}} und die \Autor{Software2007b}, die sich selbst als "`The XML Company"' bezeichnet, erweiterte k�rzlich noch einmal ihr bereits durchg�ngig an der SOA orientiertes Produktportfolio durch den Kauf des amerikanischen Unternehmens webMethods um L�sungen zur Unterst�tzung von Gesch�ftsprozessen.\footnote{\Vgl\Zitat{Software2007b}} In einem Atemzug mit der SOA werden h�ufig Webservices genannt, da sie durch ihre hohe Plattformunabh�ngigkeit und den Einsatz von Internettechnologie oftmals als Referenzimplementierung f�r die Services in einer SOA angef�hrt werden. Doch welche Vorteile bietet der Einsatz von Webservices in Unternehmen? K�nnen mit ihnen tats�chlich flexiblere Softwaresysteme entwickelt werden? Und wie einfach ist die Implementierung von Webservices auf unterschiedlichen Plattformen? Diesen Fragen wird sich der Autor in der vorliegenden Arbeit widmen.

Wie bereits in Kapitel \ref{cha:Einleitung} auf Seite \pageref{cha:Einleitung} erw�hnt, ist zur Unterst�tzung von Gesch�ftsprozessen der Einsatz von Informationstechnologie notwendig. Der Autor verfolgt mit dieser Arbeit das Ziel, einen Gesch�ftsprozess \todo{Was ist ein Gesch�ftsprozess?} mit Hilfe von Webservices zu optimieren. Hierzu wird er in diesem Kapitel eine Einf�hrung in das Thema Webservices und die damit in Zusammenhang stehenden Technologien geben, und auch auf m�gliche Einsatzbereiche von Webservices im Rahmen der Gesch�ftsprozessoptimierung eingehen. Tabelle \ref{tab:ElementeDerEreignisgesteuertenProzesskette} auf Seite \pageref{tab:ElementeDerEreignisgesteuertenProzesskette} zeigt ganz tolle Sachen.

Ich empfehle allen Softwareentwicklern die Lekt�re von \Zitat{Goodliffe2007}.

\section{Definitionen}
Die Service-orientierte Architektur ist ein Ansatz der Softwareentwicklung, der sich stark am Konzept der Gesch�ftsprozesse orientiert und mit Hilfe von Webservices implementiert werden kann. In den beiden folgenden Kapiteln werden beide Begriffe eingehend erl�utert, worauf in Kapitel \ref{cha:Fazit} die f�r die Umsetzung von Webservices ben�tigten Technologien vorgestellt werden.

\section{Service-orientierte Architektur}
\Autor{OASIS2007}\footnote{Die \NeuerBegriff{Organization for the Advancement of Structured Information Standards} ist nach \Zitat{OASIS2007} ein internationales Konsortium aus �ber 600 Organisationen, das sich der Entwicklung von E-Business-Standards verschrieben hat. Mitglieder sind \zB IBM, SAP und Sun.} definiert den Begriff \NeuerBegriff{Service-orientierte Architektur} (SOA) wie folgt:
\begin{quote}
"`\textbf{Service Oriented Architecture} [\ldots] is a paradigm for organizing and utilizing distributed \textbf{capabilities} that may be under the control of different ownership domains."'\footnote{\Zitat[S.~8]{OASIS2006a}}
\end{quote}
Diese bewusst allgemein gehaltene Definition stammt aus dem Referenzmodell der SOA aus dem Jahr 2006. Dieses Modell wurde mit dem Ziel entwickelt, ein einheitliches Verst�ndnis des Begriffs SOA und des verwendeten Vokabulars zu schaffen, und sollte die zahlreichen bis dato vorhandenen, teils widerspr�chlichen Definitionen abl�sen.\footnote{\Vgl\Zitat[S.~4]{OASIS2006a}} Dabei wird zun�chst noch kein Bezug zur Informationstechnologie hergestellt, sondern allgemein von F�higkeiten gesprochen, die Personen, Unternehmen, aber eben auch Computer besitzen und evtl. Anderen anbieten, um Probleme zu l�sen. Als Beispiel wird ein Energieversorger angef�hrt, der Haushalten seine F�higkeit Strom zu erzeugen anbietet.\footnote{\Vgl\Zitat[S.~8f.]{OASIS2006a}}

\chapter{Bilder und Listings}

In den folgenden drei Kapiteln wird der Autor eine einfache Webservice-Umgebung aufbauen, um zu zeigen, wie Webservices in der Praxis angeboten, konsumiert und orchestriert werden k�nnen. Hierzu verwendet er ausschlie�lich Open-Source-Software, im Speziellen \NeuerBegriff{Apache Tomcat}\footnote{Website: \url{http://tomcat.apache.org/}} als Servlet-Engine, \NeuerBegriff{Apache Axis2}\footnote{Website: \url{http://ws.apache.org/axis2/}} als SOAP-Engine und \NeuerBegriff{ActiveBPEL}\footnote{Website: \url{http://www.activebpel.org/}} als Workflow-System. Die Installation und Konfiguration der ben�tigten Anwendungen wird in Kapitel \ref{sec:Werkzeuge} beschrieben. Die komplette Umgebung inkl. der vom Autor erstellten Webservices befindet sich als virtuelle Maschine auf der dieser Arbeit beigelegten DVD. Im Folgenden wird der DNS-Name \Code{linux-ws} als Bezeichnung f�r den Webservice-Server verwendet.

\section{Anbieten eines Webservice}
\label{sec:AnbietenEinesWebservices}
Mit Hilfe von Apache Axis2 k�nnen Webservices sehr einfach auf Basis von normalen Java-Klassen angeboten werden. Es ist lediglich eine zus�tzliche XML-Datei namens \Datei{META-INF/services.xml} n�tig, in der die zu ver�ffentlichenden Klassen und Methoden beschrieben werden. \abbildung{HelloWorldStruktur} zeigt die Struktur eines einfachen \Webservice{HelloWorld}-Webservice.

\begin{figure}[htb]
\centering
\includegraphics[width=0.3\textwidth]{HelloWorldStruktur.jpg}
\caption{\Webservice{HelloWorld}-Webservice: Dateistruktur}
\label{fig:HelloWorldStruktur}
\end{figure}

Die Klasse \Code{HelloWorld} besitzt nur die Methode \Code{SayHello}, die den \Datentyp{String} \Code{Hello World!} zur�ckgibt. Sie wird in Listing \ref{lst:HelloWorldJava} gezeigt. 

\newpage
\lstset{language=Java, basicstyle=\footnotesize, showstringspaces=false, tabsize=2}
\lstinputlisting[label=lst:HelloWorldJava,caption=\Webservice{HelloWorld}-Webservice: Java-Klasse \Code{HelloWorld}]{DVD/Listings/HelloWorld/HelloWorld.java}

\section{Netzwerkverkehr beim Aufruf von \Webservice{PersonFactory}}

\subsection{SOAP-Request}
Listing \ref{lst:SOAPRequest} zeigt die mitgeschnittene SOAP-Anfrage per HTTP an den Webservice \Webservice{PersonFactory}. Wie am Ende von Kapitel \ref{cha:Einleitung} beschrieben, wird die eigentliche SOAP-Nachricht mittels des HTTP-\Eingabe{POST}-Befehls (Zeile 1) an den Webservice unter der angegebenen URL (Zeile 1) auf dem Server (Zeile 5) geschickt. In Zeile 3 wird �ber den Befehl \Eingabe{SOAPAction} �bermittelt, welche Funktion des Webservice (in diesem Fall \Code{CreatePerson}) aufgerufen werden soll. Die XML-Nutzlast (Zeilen 8--18) besteht dann aus einer einfachen SOAP-Nachricht aus \XMLElement{Envelope}, \XMLElement{Header} und \XMLElement{Body}, die einen RPC durchf�hrt. Die aufzurufenden Funktion wird noch einmal im SOAP-\XMLElement{Body} in Zeile 15 definiert.


\lstset{language=XML, basicstyle=\footnotesize, showstringspaces=false, tabsize=2}
\lstinputlisting[label=lst:SOAPRequest,caption=SOAP-Request an \Webservice{PersonFactory} per HTTP]{DVD/Listings/PersonFactorySOAPRequest.txt}

\subsection{SOAP-Response}
Die Antwort des \Webservice{PersonFactory}-Webservice zeigt Listing \ref{lst:SOAPResponse}. Sie beginnt in Zeile 1 mit dem HTTP-Statuscode 200, der die Anfrage als erfolgreich kennzeichnet. Die eigentliche Nutzlast in Form von XML-Daten (Zeile 3) folgt dann ab Zeile 7. Sie besteht aus dem Element \XMLElement{Person} und seinen Unterelementen, umschlossen vom Element \XMLElement{CreatePersonRepsonse}, das die Antwort-Nachricht aus der WSDL repr�sentiert.

\lstset{language=XML, basicstyle=\footnotesize, showstringspaces=false, tabsize=2}
\lstinputlisting[label=lst:SOAPResponse,caption=SOAP-Response von \Webservice{PersonFactory} per HTTP]{DVD/Listings/PersonFactorySOAPResponse.txt}

\chapter{Aufz�hlungen und Tabellen}

Eine normale Punktliste:
\begin{itemize}
\item Lorem ipsum dolor sit amet, consectetuer adipiscing elit. Nulla ac ipsum a metus viverra tempor. 
\item Nunc sem. Nulla nec urna eu nibh vehicula convallis. Integer ac turpis. Donec mauris enim, dignissim quis, scelerisque ac, rhoncus id, sapien. 
\item Donec turpis felis, cursus in, varius vitae, mollis ac, lorem. Integer a dui sit amet eros nonummy aliquet. Donec egestas adipiscing tellus. Nulla iaculis. 
\item Aliquam erat volutpat. Curabitur posuere, eros vitae accumsan semper, risus erat viverra erat, eu vehicula mi leo at elit. Fusce luctus. Fusce vehicula pretium diam. Nunc sed arcu ut erat suscipit fermentum.
\end{itemize}

Eine nummerierte Liste:
\begin{enumerate}
\item Lorem ipsum dolor sit amet, consectetuer adipiscing elit. Nulla ac ipsum a metus viverra tempor. 
\item Nunc sem. Nulla nec urna eu nibh vehicula convallis. Integer ac turpis. Donec mauris enim, dignissim quis, scelerisque ac, rhoncus id, sapien. 
\item Donec turpis felis, cursus in, varius vitae, mollis ac, lorem. Integer a dui sit amet eros nonummy aliquet. Donec egestas adipiscing tellus. Nulla iaculis. 
\item Aliquam erat volutpat. Curabitur posuere, eros vitae accumsan semper, risus erat viverra erat, eu vehicula mi leo at elit. Fusce luctus. Fusce vehicula pretium diam. Nunc sed arcu ut erat suscipit fermentum.
\end{enumerate}

\section{Vom Autor verwendete Software}
\label{sec:Werkzeuge}
Im Folgenden werden die Programme vorgestellt, die der Autor zum Erstellen dieser Arbeit und vor allem zur Entwicklung der Webservices verwendet hat. Soweit es m�glich war, wurden Open-Source-Programme eingesetzt.

\begin{itemize}
\itemd{Microsoft Visio}{Die EPKs der BAP wurden mit Microsoft Visio erstellt. Der Autor hat zwar verschiedene Open-Source-Programme\footnote{Dia, OpenOffice Draw und die EPC Tools.} ausprobiert, mit denen EPKs erstellt werden k�nnten, die grafischen Ergebnisse waren aber nicht zufriedenstellend. Die Symbole von Visio sehen den "`originalen"' ARIS-Symbolen am �hnlichsten und k�nnen dar�ber hinaus mit zus�tzlichen Informationen wie Dauer und Kosten versehen werden.}
\itemd{PSPad}{F�r die Bearbeitung von verschiedenen (Text-)Dateien wurde der Texteditor PSPad verwendet. Mit diesem konnten \zB auch die regul�ren Ausdr�cke f�r die XML-Schemas entwickelt werden. Website: \url{http://www.pspad.com/}}
\itemd{Eclipse}{Sowohl der ActiveBPEL Designer als auch die EntireX Workbench sind Plugins f�r die IDE Eclipse. Auch zur Java- und PHP-Entwicklung wurde dieses Werkzeug verwendet. Website: \url{http://www.eclipse.org/}}
\itemd{XML Copy Editor}{F�r die Entwicklung der XML-Schemas und die Bearbeitung von XML-Dateien wurde der XML Copy Editor eingesetzt. Mit diesem k�nnen \ua XML-Dateien auf Wohlgeformtheit gepr�ft und gegen ihr Schema validiert werden. Website: \url{http://xml-copy-editor.sourceforge.net/}}
\itemd{soapUI}{
Mit soapUI k�nnen Webservices getestet werden, ohne einen Client zu programmieren. Die SOAP-Anfragen werden automatisch anhand der WSDL generiert und die Antworten k�nnen gegen die WSDL-Datei validiert werden. Website: \url{http://www.soapui.org/}}
\itemd{Ethereal}{
Die Netzwerkkommunikation beim Aufrufen der Webservices wurde mit Ethereal, einem umfangreichen Werkzeug zur Analyse des Netzwerkverkehrs, mitgeschnitten. Website: \url{http://www.ethereal.com/}}
\itemd{\LaTeX}{
Diese Arbeit wurde mit {\LaTeX} geschrieben. Als Distribution wurde MiKTeX verwendet und als Editor der LaTeX Editor. Websites: \url{http://miktex.org/}, \url{http://www.latexeditor.org/}}
\end{itemize}

\section{Elemente der Ereignisgesteuerten Prozesskette}

\begin{longtable}{|m{10cm}|m{3cm}|}
\caption{Elemente der Ereignisgesteuerten Prozesskette} \\
\hline
\label{tab:ElementeDerEreignisgesteuertenProzesskette}
\textbf{Element} & \textbf{Symbol}\\
\hline
\textbf{Funktion} 

Funktionen beschreiben T�tigkeiten, die im Verlauf des Gesch�ftsprozesses anfallen. Sie k�nnen von Mitarbeitern oder einem Informationssystem durchgef�hrt werden und ben�tigen evtl. Ressourcen, die ihnen zugewiesen werden. 

Beispiele: \textit{Auftrag anlegen}, \textit{Rechnung schreiben}, \textit{Konto abschlie�en} & 
\includegraphics[width=3cm]{EPK-Funktion.jpg} \\
\hline
\textbf{Ereignis} 

Ereignisse sind betriebswirtschaftlich relevante Ereignisse, die den Gesch�ftsprozess in irgendeiner Weise steuern oder beeinflussen. Ereignisse sind immer Ausl�ser oder Ergebnisse von Funktionen. Ein Gesch�ftsprozess beginnt und endet stets mit einem Ereignis. 

Beispiele: \textit{Auftrag eingetroffen}, \textit{�berweisung get�tigt}, \textit{Rechnung erstellt} & 
\includegraphics[width=3cm]{EPK-Ereignis.jpg} \\
\hline
\textbf{Operatoren} 

Operatoren steuern den Kontrollfluss eines Gesch�ftsprozesses. Sie machen \zB deutlich, dass eine Funktion mehrere Ereignisse ausl�st, oder zeigen alternative Vorgehensweisen an. Es gibt drei Operatoren (v.\,l.\,n.\,r.\,): UND, ODER und XODER (exklusives ODER). & 
\includegraphics[width=3cm]{EPK-Operatoren.jpg} \\
\hline
\textbf{Organisationseinheit} 

Organisationseinheiten werden Funktionen zugeordnet und beschreiben, wo die Funktionen ausgef�hrt werden bzw. wer sie ausf�hrt. Die Bezeichnung der Symbole enth�lt zus�tzlich zur Abteilung noch die Namen der Mitarbeiter.

Beispiele: \textit{Vertrieb}, \textit{Personal}, \textit{Produktion} & 
\includegraphics[width=3cm]{EPK-Organisationseinheit.jpg} \\
\hline
\textbf{Informationsobjekt} 

Auch Informationsobjekte werden Funktionen zugewiesen und beschreiben die von diesen ben�tigten oder erstellten Informationen. Dabei sind s�mtliche Formen von Informationen auf verschiedenen Datentr�gern m�glich und nicht etwa nur digitale Daten. Die Bezeichnung der Symbole enth�lt zus�tzlich das Informationssystem, aus dem die Informationen stammen.

Beispiele: \textit{Kundendatenbank}, \textit{Versicherungsantrag}, \textit{Rechnung} & 
\includegraphics[width=3cm]{EPK-Informationen.jpg} \\
\hline
\textbf{Prozesswegweiser}

Mit Prozesswegweisern werden Prozesse, die in anderen EPKs beschrieben sind, referenziert. So k�nnen \zB un�bersichtliche Prozesse in Teilprozesse gegliedert und h�ufig verwendete Prozesse an zentraler Stelle modelliert werden. Prozesswegweiser stehen in einer EPK immer anstelle von Funktionen. & 
\includegraphics[width=3cm]{EPK-Prozesspfad.jpg} \\
\hline
\end{longtable}

\chapter{Fazit und kritische Bewertung}
\label{cha:Fazit}
Lorem ipsum dolor sit amet, consectetuer adipiscing elit. Nulla ac ipsum a metus viverra tempor. Nunc sem. Nulla nec urna eu nibh vehicula convallis. Integer ac turpis. Donec mauris enim, dignissim quis, scelerisque ac, rhoncus id, sapien. Donec turpis felis, cursus in, varius vitae, mollis ac, lorem. Integer a dui sit amet eros nonummy aliquet. Donec egestas adipiscing tellus. Nulla iaculis. Aliquam erat volutpat. Curabitur posuere, eros vitae accumsan semper, risus erat viverra erat, eu vehicula mi leo at elit. Fusce luctus. Fusce vehicula pretium diam. Nunc sed arcu ut erat suscipit fermentum.

Proin id magna eu sem tincidunt feugiat. Sed tincidunt massa sed eros. Fusce condimentum eros et lectus. Pellentesque lectus tortor, mattis in, dapibus a, lobortis ut, justo. Sed id dolor ut nibh varius ultrices. Quisque tincidunt nisl vel nibh. Suspendisse sodales massa non magna. In porttitor augue nonummy nunc. Nam quis enim quis ante dapibus interdum. Morbi nec neque. Fusce pharetra consectetuer magna. Etiam laoreet, augue nec lacinia ornare, risus purus lobortis erat, eu consequat urna orci vel arcu. Integer cursus, augue sed tempor dapibus, erat tortor rutrum elit, sit amet fermentum purus neque vitae tortor. Donec vulputate, ipsum vel viverra pretium, purus orci mattis nulla, nec tincidunt leo metus sed ipsum. Fusce eget lectus sed lectus molestie tincidunt. Etiam tincidunt urna eget tortor.

Sed sit amet magna at lectus interdum blandit. Proin vitae metus eget leo bibendum ornare. Morbi sit amet nisl ac odio accumsan laoreet. Etiam luctus massa vel enim. Vestibulum nulla tellus, viverra at, malesuada vel, volutpat quis, lorem. Vestibulum quis nulla. Curabitur neque nibh, bibendum vel, eleifend sit amet, euismod at, leo. Duis auctor lobortis justo. Donec in tortor vel nibh rutrum pellentesque. Curabitur blandit pede quis neque. Nam sem eros, ornare a, pretium eget, condimentum sed, leo. Curabitur orci felis, elementum eget, aliquet vel, porta id, velit. Etiam justo neque, rhoncus quis, elementum vel, auctor vitae, urna.




% Literaturverzeichnis ---------------------------------------------------------
%   Das Literaturverzeichnis wird aus der BibTeX-Datenbank "Bibliographie.bib"
%   erstellt.
% ------------------------------------------------------------------------------
\bibliography{Bibliographie} % Aufruf: bibtex Masterarbeit
\bibliographystyle{natdin} % DIN-Stil des Literaturverzeichnisses

\include{Erklaerung} % Selbst�ndigkeitserkl�rung

% Anhang -----------------------------------------------------------------------
%   Die Inhalte des Anhangs werden analog zu den Kapiteln inkludiert.
%   Dies geschieht in der Datei "Anhang.tex".
% ------------------------------------------------------------------------------
\begin{appendix}
    \clearpage
    \pagenumbering{roman}
    \chapter{Anhang}
    \label{sec:Anhang}
    % Rand der Aufz�hlungen in Tabellen anpassen
    \setdefaultleftmargin{1em}{}{}{}{}{}
    \input{Anhang}
\end{appendix}

% Index ------------------------------------------------------------------------
%   Zum Erstellen eines Index, die folgende Zeile auskommentieren.
% ------------------------------------------------------------------------------
%\printindex

\end{document}
